%----------------------------------------------------------------------------------------
%	TITLE PAGE
%----------------------------------------------------------------------------------------
\documentclass[aspectratio=169]{beamer}
% use 16:9 in new slides!

\usepackage{tikz}
\usepackage{listings}
\usepackage{xcolor}
\usepackage{graphicx}
\usepackage{float}
\usepackage{tikz}
\usetikzlibrary{trees}
\usepackage{hyperref}
\usepackage{array}
\usepackage[skins]{tcolorbox}
\def\bpr#1#2{
\begin{tcolorbox}
[boxsep=.15cm,left=.2cm,right=.2cm,oversize,boxrule=0mm,
colback=green!55!blue!20!white!60,
colframe=red!50!yellow!50!white,
colbacktitle=blue!50, 
coltitle=black,enhanced,drop fuzzy shadow,
fonttitle=\bfseries,title=#1]
#2
\end{tcolorbox}
}

\def\bdef#1#2{
\begin{tcolorbox}
[boxsep=.15cm,left=.2cm,right=.2cm,oversize,boxrule=0mm,
colback=white!60,
colframe=red!50!yellow!50!white,
colbacktitle=green!50!yellow!60!gray, 
coltitle=black,enhanced,drop fuzzy shadow,
fonttitle=\bfseries,title=#1]
#2
\end{tcolorbox}
}

\def\bthe#1#2{
\begin{tcolorbox}
[boxsep=.15cm,left=.2cm,right=.2cm,oversize,boxrule=0mm,
colback=white!60,
colframe=red!50!yellow!50!white,
colbacktitle=red!52, 
coltitle=black,enhanced,drop fuzzy shadow,
fonttitle=\bfseries,title=#1]
#2
\end{tcolorbox}
}

\def\tcb#1{
\begin{tcolorbox}
[boxsep=.15cm,left=.2cm,right=.2cm,oversize,boxrule=0mm,
colback=white!60,
colframe=red!50!yellow!50!white,
colbacktitle=red!50!yellow!50!white, coltitle=black,enhanced,drop fuzzy shadow,]
#1
\end{tcolorbox}
}



\definecolor{UMBlue}{RGB}{5,32,103}
\definecolor{UMYellow}{RGB}{255,216,0}
%\usetheme{Madrid}
\setbeamertemplate{headline}{
  \leavevmode%
  \begin{minipage}{0.75\paperwidth}
  \vspace{1.2ex}\hspace{-0.245\paperwidth}
  \resizebox{\paperwidth}{3ex}{
    \tikz{
      \fill [color=UMBlue] (0,0) rectangle (10, 0.13);
      \fill [color=UMYellow] (0,-0.1) rectangle (10, -0.18);
    }
  }
  \begin{beamercolorbox}[wd=\linewidth,ht=2.5ex,dp=1.125ex]{section}
    \insertsubsectionnavigationhorizontal{\linewidth}{}{Slide \insertpagenumber}
  \end{beamercolorbox}
  \end{minipage}
  \begin{minipage}{0.23\paperwidth}
  \hspace{0.2em}
  \includegraphics[width=\linewidth]{assets/logo.png}
  \end{minipage}
}

\setbeamercolor{title}{fg=UMBlue}
\setbeamercolor{frametitle}{fg=UMBlue}
\setbeamercolor{structure}{fg=UMBlue}

\title[Course number]{Basic Git Workshop} 
\author[]{TechJI}
\institute[UMJI-SJTU]
{
	University of Michigan - Shanghai Jiaotong University
	\\\medskip
	Joint Institute
}
\date{2023.10.8}
%----------------------------------------------------------------------------------------
%	Highlight the title of the current section
%----------------------------------------------------------------------------------------
\AtBeginSection[]
{
  \begin{frame}
    \frametitle{Table of Contents}
    \tableofcontents[currentsection]
  \end{frame}
}



\begin{document}
% insert title page---------------------------
\maketitle
%insert contents------------------------------
\begin{frame}
  \frametitle{Table of Contents}
  \tableofcontents
\end{frame}

\begin{frame}
  \frametitle{Before we start}
  \begin{itemize}
      \item This is \textbf{not} a Linux workshop
          (although I encourage you to use it).
      \item This is \textbf{not} a Vim workshop
          (although I encourage you to use it).
      \item This is \textbf{not} a Bash workshop either.
      \item We are organizing this workshop primarily because many students encountered difficulties when using Git in courses ENGR151 and SilverFOCS(VG100).
      \item The most important part in this workshop is $lazygit$.
      \item The target audience for this workshop are those who are unfamiliar with Git or do not understand the working principles of Git.
  \end{itemize}
\end{frame}

\section{Introduction}
\subsection{Git}
% insert a sample frame without animation--------------------------------
\begin{frame}
  \frametitle{What is Git?}
  Git is a free and open source distributed version control system designed to handle everything from small to very large projects with speed and efficiency.
\end{frame}

\begin{frame}
  \frametitle{When to use Git?}

  \begin{itemize}
    \item Writing code
    \item Managing docs
    \item Managing projects
    \item Version control
    \item Team corporation
  \end{itemize}

\end{frame}

\begin{frame}
  \frametitle{Git Installation}
  You can go to the official website to download the installation package and refer to $Installation\_git$ file in the repo.
\end{frame}

\begin{frame}
  \frametitle{Where to use Git?}
  We first introduce shell commands, then we will use Git in the shell.
\end{frame}

\subsection{Shell}
\begin{frame}
  \frametitle{What is Shell?}
  Shell is a program that takes commands from the keyboard and gives them to the operating system to perform. It is also referred to as a command-line interpreter or CLI.
  \\
  Common shells:
  \begin{itemize}
    \item Bash
    \item Zsh
    \item etc
  \end{itemize}
\end{frame}

\begin{frame}
  \frametitle{A brief history of bash}
  \begin{figure}[h]
    \centering
    \includegraphics[height=3cm]{./assets/bash_logo.png}
  \end{figure}
  \begin{itemize}
    \item Born: 1989
    \item Probably played Pokémon on the Game Boy
    \item Is an umbrella term for zsh, fish, …
    \item Runs on Unix-like environments
  \end{itemize}
\end{frame}

\begin{frame}
  \frametitle{A brief history of Unix}
  \begin{figure}[h]
    \centering
    \includegraphics[height=3cm]{./assets/unix_logo.png}
  \end{figure}
  \begin{itemize}
    \item Born: 1969
    \item Probably listened to Michael Jackson
    \item Gave rise to Linux, BSD, and Mac OS
    \item We call them ``Unix-like''
  \end{itemize}
\end{frame}

\begin{frame}
  \frametitle{Unix: The Good Part}
  The Unix philosophy (paraphrased):
  \begin{itemize}
    \item Store data in plain text
    \item Hierarchical file system
    \item Everything is a file
    \item One tool does one thing
    \item Tools together strong
  \end{itemize}
  \begin{block}{Quote}
    The power of a system comes more from the relationships among programs than
    from the programs themselves.
    \begin{flushright}
      — Brian Kernighan and Rob Pike
      \footnote{The UNIX Programming Environment. 1984. viii}
    \end{flushright}
  \end{block}
\end{frame}

\begin{frame}
  \frametitle{How to open a Shell?}
  \begin{itemize}
    \item Windows: cmd, powershell, Git Bash
    \item Mac: Terminal
    \item Linux: Terminal
  \end{itemize}
\end{frame}

\section{Basic Commands}

\begin{frame}[fragile]
  \frametitle{Files}
  Each of these is a different \textbf{file}:
  \begin{itemize}
    \item \tt{a}
    \item \tt{.a} (Hidden)
    \item \tt{a.txt}
    \item \tt{A.txt}
    \item \tt{A.TXT}
  \end{itemize}

  \begin{block}{Note}
    The dot and suffix are part of the filename. Windows users please turn on \textbf{show file extensions}.
    \newline \newline
    \textbf{Avoid spaces and special characters} (except \verb|._-|).
    If you have to, surround filename in quotes:
    \tt{`Lab Report (3) final FINAL-1.docx'}
  \end{block}
\end{frame}

\begin{frame}
  \frametitle{Directories}
  Each of these is a \textbf{directory} (``dir'' for short):
  \begin{itemize}
    \item \tt{hteam-10086/}
    \item \tt{hteam-10086/h1/}
    \item \tt{hteam-10086/.gitea/} (Hidden dir)
  \end{itemize}

  \begin{block}{Convention}
    For clarity, we add a slash (\tt{/}) to the end of a directory in the
    slides. However, in reality it often makes no difference.
  \end{block}
\end{frame}

\begin{frame}
  \frametitle{\tt{cd, pwd}: Changing directory}
  \begin{itemize}
    \item \tt{cd hteam-10086/}
    \item \tt{pwd}
  \end{itemize}
  \begin{block}{Explanation}
    \begin{itemize}
      \item \tt{cd}: "change directory"
      \item \tt{pwd}: "print working directory"
      \item \tt{../} means "parent directory"
      \item \tt{./} means "current directory"
      \item \tt{\~{}} means "home directory"
    \end{itemize}
  \end{block}
\end{frame}

\begin{frame}
  \frametitle{Paths}
  File $\cup$ directory = \textbf{path}.
  \footnote{At least in the scope of this workshop.}
  \newline \newline

  No paths under the same directory can bear the same name.
  These \textbf{cannot} coexist:
  \begin{itemize}
    \item \tt{hteam-10086/h1/}, a directory
    \item \tt{hteam-10086/h1/ex1.m}, a regular file
  \end{itemize}
\end{frame}

\begin{frame}
  \frametitle{Absolute \& relative paths}
  \begin{itemize}
    \item Paths beginning with \tt{/} are absolute: \tt{/usr/bin/cat}
    \item Otherwise it is relative: \tt{hteam-10086}
  \end{itemize}

  If you know where you are, you can convert a relative path to an absolute one.
\end{frame}

\begin{frame}
  \frametitle{\tt{ls}: Listing directories}
  \begin{itemize}
    \item ls
    \item ls -a
    \item ls -l
    \item ls -la
  \end{itemize}
  \begin{block}{Explanation}
      \begin{itemize}
          \item \tt{ls}: ``list''
          \item \tt{-a} is short for \tt{--all}
          \item \tt{-l} enables long listing format
          \item \tt{-la} = \tt{-l} + \tt{-a}
      \end{itemize}
  \end{block}
\end{frame}

\begin{frame}
  \frametitle{More...}
  In today's workshop, you need to know how to get into your repository.\\
  More usage of Bash and Shell will be explored in the spring semester's Bash Workshop. You can refer to the cheatsheet in this repository.
\end{frame}

\section{Basic Git}
\begin{frame}
  \frametitle{Some basic git commands}
  Tell Git who you are:
  \begin{itemize}
    \item git config --global user.name "Your Name"
    \item git config --global user.email "Your Email"
  \end{itemize}
\end{frame}

\begin{frame}
  \frametitle{Some basic git commands}
  How to use "git add":
  \begin{itemize}
    \item git add $file$
    \item git add *
    \item git add .
    \item git add -A
  \end{itemize}
  use "git status" to check the status of your repo.

\end{frame}

\begin{frame}
  \frametitle{Some basic git commands}
  If you want some files never be tracked by git, you can create a file named ".gitignore" in your repo, and write the file names in it.
 
\end{frame}

\begin{frame}
  \frametitle{Some basic git commands}
  How to use "git commit":
  \begin{itemize}
    \item git commit -m "commit message"
  \end{itemize}
  If you only type: 
  \begin{itemize}
    \item git commit
  \end{itemize}
  then you will enter vim /other default editor to write your commit message.\\
  \href{https://www.conventionalcommits.org/en/v1.0.0/}{How to write commit message?}
\end{frame}

\begin{frame}
  \frametitle{Some basic git commands}
  How to use "git push":
  \begin{itemize}
    \item git push origin $branch$
    \item git push origin master
    \item git push
  \end{itemize}
\end{frame}

\begin{frame}
  \frametitle{Some basic git commands}
  How to use "git rm":
  \begin{itemize}
    \item git rm --cached $file$
  \end{itemize}
  Then, git will stop tracking this file, but the file still exists in your repo.
\end{frame}

\begin{frame}
  \frametitle{Some basic git commands}
  How to delte a file in your remote repo:
  \begin{itemize}
    \item git rm $file$
    \item git commit -m "remove file"
    \item git push
  \end{itemize}
\end{frame}

\begin{frame}
  \frametitle{Some basic git commands}
  How to use "git pull":
  \begin{itemize}
    \item git pull
  \end{itemize}
\end{frame}

\begin{frame}
  \frametitle{Some basic git commands}
  How to use branch in git:
  \begin{itemize}
    \item create new branch(based on current branch): git checkout -b $branch$
    \item go to other branch: git checkout $branch$
    \item delete branch: git branch -d $branch$
  \end{itemize}
\end{frame}

\begin{frame}
  \frametitle{Some basic git commands}
  How to use "git merge":
  For example, we are now on branch "master", and we want to merge branch "dev" to "master":
  \begin{itemize}
    \item git checkout dev
    \item git pull
    \item git checkout master
    \item git merge dev
    \item git push
  \end{itemize}
\end{frame}

\begin{frame}
  \frametitle{Merge Conflict}
  Sometimes, you will encounter merge conflicts in the process of merge.\\
  What is merge conflict?\\
  \begin{itemize}
    \item A merge conflict is an event that occurs when Git is unable to automatically resolve differences in code between two commits.\\
  \end{itemize}
  What causes merge conflict?\\
  \begin{itemize}
    \item When two people make changes to the same line of a file.
  \end{itemize}
\end{frame}

\begin{frame}
  \frametitle{Merge Conflict}
  In most cases, Git will attempt to auto-merge first. If a merge conflict occurs, Git will inform you which file has a conflict, and you'll have to manually edit that file.\\
\end{frame}


\begin{frame}
  \frametitle{How to solve Merge Conflict?}
  \begin{itemize}
    \item Open the file and look for Git's conflict markers.\\
    \item Resolve the conflict by choosing which code to keep.\\
    \item Add your changes (git add .) and commit them (git commit -m "merged branch").\\
    \item Now, you can push your changes to the remote repository.\\
    \item git push\\
    \item If you want to abort the merge, you can use "git merge --abort"\\
    \item If you want to abort the merge and go back to the state before the merge, you can use "git reset --hard HEAD"\\
  \end{itemize}
\end{frame}

  % Suggestions: You can use "lazygit" or search for "Git merge tool" online.
\begin{frame}
  \frametitle{Some basic git commands}
  How to use "git revert":
  \begin{itemize}
    \item git revert $commit-SHA$
    \item git revert HEAD
  \end{itemize}
\end{frame}

\section{Lazygit}
\begin{frame}{Lazygit}
  \href{https://github.com/jesseduffield/lazygit}{Lazygit} is a powerful Git frontend that integrates many git commands. It is written in Go and is cross-platform.

  \pause

  Why lazygit?
  \begin{itemize}
      \item Fast, TUI
      \item Easy to use
      \item Powerful
  \end{itemize}

  Other alternatives are \href{https://github.com/Extrawurst/gitui}{Gitui} which is written in Rust.
\end{frame}

\begin{frame}{Tasks\footnote{Operations with * will rewrite history.}}
          \begin{itemize}[<+->]
              \item Install lazygit. You may refer to $Installation\_Lazygit$ in repo.(10 mins)
              \item Clone your repository.
              \item cd into your repo and type "lazygit" in terminal
              \item Navigate the interface (h/j/k/l/[/]/arrow keys).
              \item Stage/Unstage files (a/A).
              \item Commit changes (c).
              \item Push files (P).
              \item Create/delete branches (n/d).
              \item Checkout branches (space).
              \item Merge branches (M).
          \end{itemize}
\end{frame}

\begin{frame}
  \frametitle{More...}
  More usage of Git will be explored in the next year's Advanced Git Workshop. You can refer to the cheatsheet in this repository.
\end{frame}


\begin{frame}{References}
  \begin{itemize}
      \item Pro Git. \href{https://git-scm.com/book/en/v2}{Git - Book}
      \item fakefred/bash-workshop
      \item linsyking/git-wksp
  \end{itemize}
\end{frame}

\begin{frame}{Reading Materials}
  \begin{itemize}
      \item (Highly recommend)TheCW-Git \emph{Bilibili}. \href{https://www.bilibili.com/video/BV1Yx411f7Cu}{BV1Yx411f7Cu}
      \item (Highly recommend)TheCW-Lazygit \emph{Bilibili}. \href{https://www.bilibili.com/video/BV1gV411k7fC}{BV1gV411k7fC}
      \item Pro Git. \href{https://git-scm.com/book/en/v2}{Git - Book}
      \item \href{https://missing.csail.mit.edu/2020/course-shell/}{MIT Course}
      \item \href{https://learngitbranching.js.org/?locale=zh_CN}{Learn git online}
      \item \href{https://www.liaoxuefeng.com/wiki/896043488029600}{Liao Xuefeng's Git Tutorial(Chinese)}
  \end{itemize}
\end{frame}

\begin{frame}
  \Huge{\textcolor{UMBlue}{Thanks for listening!}}
\end{frame}



\end{document}
